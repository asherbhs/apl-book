%% Generated by Sphinx.
\def\sphinxdocclass{jupyterBook}
\documentclass[letterpaper,10pt,english]{jupyterBook}
\ifdefined\pdfpxdimen
   \let\sphinxpxdimen\pdfpxdimen\else\newdimen\sphinxpxdimen
\fi \sphinxpxdimen=.75bp\relax
\ifdefined\pdfimageresolution
    \pdfimageresolution= \numexpr \dimexpr1in\relax/\sphinxpxdimen\relax
\fi
%% let collapsible pdf bookmarks panel have high depth per default
\PassOptionsToPackage{bookmarksdepth=5}{hyperref}
%% turn off hyperref patch of \index as sphinx.xdy xindy module takes care of
%% suitable \hyperpage mark-up, working around hyperref-xindy incompatibility
\PassOptionsToPackage{hyperindex=false}{hyperref}
%% memoir class requires extra handling
\makeatletter\@ifclassloaded{memoir}
{\ifdefined\memhyperindexfalse\memhyperindexfalse\fi}{}\makeatother

\PassOptionsToPackage{warn}{textcomp}

\catcode`^^^^00a0\active\protected\def^^^^00a0{\leavevmode\nobreak\ }
\usepackage{cmap}
\usepackage{fontspec}
\defaultfontfeatures[\rmfamily,\sffamily,\ttfamily]{}
\usepackage{amsmath,amssymb,amstext}
\usepackage{polyglossia}
\setmainlanguage{english}



\setmainfont{FreeSerif}[
  Extension      = .otf,
  UprightFont    = *,
  ItalicFont     = *Italic,
  BoldFont       = *Bold,
  BoldItalicFont = *BoldItalic
]
\setsansfont{FreeSans}[
  Extension      = .otf,
  UprightFont    = *,
  ItalicFont     = *Oblique,
  BoldFont       = *Bold,
  BoldItalicFont = *BoldOblique,
]
\setmonofont{FreeMono}[
  Extension      = .otf,
  UprightFont    = *,
  ItalicFont     = *Oblique,
  BoldFont       = *Bold,
  BoldItalicFont = *BoldOblique,
]



\usepackage[Bjarne]{fncychap}
\usepackage[,numfigreset=1,mathnumfig]{sphinx}

\fvset{fontsize=\small}
\usepackage{geometry}


% Include hyperref last.
\usepackage{hyperref}
% Fix anchor placement for figures with captions.
\usepackage{hypcap}% it must be loaded after hyperref.
% Set up styles of URL: it should be placed after hyperref.
\urlstyle{same}

\addto\captionsenglish{\renewcommand{\contentsname}{Combinatorics}}

\usepackage{sphinxmessages}



        % Start of preamble defined in sphinx-jupyterbook-latex %
         \usepackage[Latin,Greek]{ucharclasses}
        \usepackage{unicode-math}
        % fixing title of the toc
        \addto\captionsenglish{\renewcommand{\contentsname}{Contents}}
        \hypersetup{
            pdfencoding=auto,
            psdextra
        }
        % End of preamble defined in sphinx-jupyterbook-latex %
        

\title{Some Mathematics in Dyalog APL}
\date{Dec 17, 2023}
\release{}
\author{Asher Harvey\sphinxhyphen{}Smith}
\newcommand{\sphinxlogo}{\vbox{}}
\renewcommand{\releasename}{}
\makeindex
\begin{document}

\pagestyle{empty}
\sphinxmaketitle
\pagestyle{plain}
\sphinxtableofcontents
\pagestyle{normal}
\phantomsection\label{\detokenize{intro::doc}}

\begin{itemize}
\item {} 
\sphinxAtStartPar
Combinatorics

\begin{itemize}
\item {} 
\sphinxAtStartPar
{\hyperref[\detokenize{combinatorics/combinatorics::doc}]{\sphinxcrossref{What is Combinatorics?}}}

\item {} 
\sphinxAtStartPar
{\hyperref[\detokenize{combinatorics/enumerative::doc}]{\sphinxcrossref{Enumerative Combinatorics and The Twelvefold Way}}}

\end{itemize}
\end{itemize}
\begin{itemize}
\item {} 
\sphinxAtStartPar
Graph Theory

\begin{itemize}
\item {} 
\sphinxAtStartPar
{\hyperref[\detokenize{graph-theory/graph-theory::doc}]{\sphinxcrossref{Graph Theory}}}

\end{itemize}
\end{itemize}

\sphinxstepscope


\part{Combinatorics}

\sphinxstepscope


\chapter{What is Combinatorics?}
\label{\detokenize{combinatorics/combinatorics:what-is-combinatorics}}\label{\detokenize{combinatorics/combinatorics::doc}}
\sphinxstepscope


\chapter{Enumerative Combinatorics and The Twelvefold Way}
\label{\detokenize{combinatorics/enumerative:enumerative-combinatorics-and-the-twelvefold-way}}\label{\detokenize{combinatorics/enumerative::doc}}
\sphinxstepscope


\part{Graph Theory}

\sphinxstepscope


\chapter{Graph Theory}
\label{\detokenize{graph-theory/graph-theory:graph-theory}}\label{\detokenize{graph-theory/graph-theory::doc}}






\renewcommand{\indexname}{Index}
\printindex
\end{document}